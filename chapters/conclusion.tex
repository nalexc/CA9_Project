\chapter{Conclusion}
The overall objective of this project was to consider several satellites flying in formation with the purpose of pointing towards a target. In order to reach this goal, two controllers were designed: one for controlling the angle between satellites using the drag force and another one for attitude control in order to be able to rotate the satellite to the desired orientation.

First, for designing a controller for the angle, the relative dynamics between two satellites are analyzed. Therefore, an LQR controller is implemented to control the angle between two satellites. The simulations showed that the LQR performed properly. Additionally, two algorithms for formation control are designed, a global and distributed algorithm. Both algorithms are working as intended, where the angles between neighbour satellites converged to the desired angle of 45$^{\circ}$, with the mention that in the distributed algorithm case the convergence rate will be slower compared with the global algorithm.

Afterwards, two control methods to obtain the desired orientation of the satellite has been implemented. The first method is a state feedback which is a linear control method. For implementing this controller the equation of motion need to be linearized. The second method is using a non-linear control method called sliding mode control. The results from sliding mode control showed that the error quaternion converged better compared to state feedback control, but in this case, the state feedback is deemed to be more suitable. Besides, the sliding mode convergence is better, but due to the fact that the control law is more complex, the satellite controler will need more computation time. 

In conclusion, some acceptance tests have been made to establish that the requirements accomplished. 