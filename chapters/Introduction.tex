\chapter{Introduction}\label{chap:Introduction}
In the last decades, the space technology is continuously growing. The reason for this is the increased deploying of satellites used in the numerous fields, in particular, telecommunications and meteorology.  \cite{SIDI}

Missions containing several satellites are commonly referred to as flying formation, which is known as a distributed satellite system. Two types of distributed space systems are identified as formations and constellations flying. 

A distributed space system is defined by NASA Goddard Space Flight Center (GSFC) as \textit{"an end-to-end system including two or more space vehicles and a cooperative infrastructure for scientific measurement, data acquisition, processing, analysis, and distribution".} \cite{SFF}

Satellite formation flying is not having a precise definition, however, the definition proposed by NASA GSFC is that \textit{"formation flight involves the use of an active control scheme to maintain the relative positions of the spacecraft "}. In contrast, a constellation is defined as \textit{"two or more spacecraft in similar orbits with no active control by either to maintain a relative position"}. \cite{SF}

Formation flying it might offer many possibilities for space exploration, such as surveillance, field measurements and atmospheric survey missions as well as on-orbit satellite inspection, maintenance, and recovery. This approach it has a few challenges which involve autonomous control of the satellites influenced by the different disturbing forces caused by solar radiation pressure, aerodynamic drag, and Earth’s oblateness effect, with a purpose of achieving it with minimum fuel consumption. Nevertheless, there is currently no formation flying satellites in orbit, however, two such missions are ESA’s "Cluster" mission and the ESA/NASA "Grace" mission, which are in development stages. \cite{SF}

The use of satellite formations is expected to rise in the next years. This makes it relevant to look at improving or adding functionalities to satellites. Based on this it has been decided to look at the case of a distributed space system consisting of a formation of eight satellites equally distributed on the orbit and analyzing the behavior between them.
 \vspace{2cm}
\section{Problem statement}
Design and simulation of a control scheme which maintain a constant angle between satellites by controlling the attitude of each agent.
\section{Use-case}\label{sec:useCase}
In this project, the concept of a formation flight of satellites will be used for the purpose of monitoring. Denmark has a small island called Greenland, where the Danish Government needs to monitor it. One method is to have a formation of satellites pointing towards Greenland when they are located above target.

One of the essentials in formation flight is choosing the number of satellites in orbit. Therefore, in order to have a continuous coverage, a distributed satellite system composed of eight satellites equally distributed are chosen, compared with two or four satellites where communication between each other will be poor.
% After the surveillance, the concept is to change the attitude and having a control of the distance between them as they are in orbit.

The task the satellite has to perform is acquiring data by flying around Greenland, using radio signals and taking pictures.

